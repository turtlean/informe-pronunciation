% !TEX TS–program = pdflatexmk
%%%%%%%%%%%%%%%%%%%%%%%%%%%%%%%%%%%%%%%%%
% Contract
% LaTeX Template
% Version 1.0 (December 8 2014)
%
% This template has been downloaded from:
% http://www.LaTeXTemplates.com
%
% Original author:
% Brandon Fryslie
% With extensive modifications by:
% Vel (vel@latextemplates.com)
%
% License:
% CC BY-NC-SA 3.0 (http://creativecommons.org/licenses/by-nc-sa/3.0/)
%
% Note:
% If you are using Apple OS X, go into structure.tex and uncomment the font
% specifications for OS X and comment out the default specifications - this will
% drastically increase how good the document looks. You will now need to
% compile with XeLaTeX.
%
%%%%%%%%%%%%%%%%%%%%%%%%%%%%%%%%%%%%%%%%%

\documentclass[a4paper,12pt]{article} % The default font size is 12pt on A4 paper, change to 'usletter' for US Letter paper and adjust margins in structure.tex
\usepackage[utf8]{inputenc}
\usepackage[pdftex]{graphicx}
% \usepackage{biblatex}
\usepackage[
backend=biber,
sorting=ynt
]{biblatex}
\addbibresource{references.bib}

\input{structure.tex} % Input the structure.tex file which specifies the document layout and style

\title{	
\normalfont \normalsize 
\textsc{Universidad de Buenos Aires \\ Facultad de Ciencias Exactas y Naturales \\ Departamento de Computaci\'on} \\ [25pt] % Your university, school and/or department name(s)
\huge Análisis de features dinámicas para sistemas de calificación de pronunciación
de lengua extranjera\\ % The assignment title
\date{\today}
\vspace{20px}
}

\author{
	Alumno: Leandro Matayoshi \\ 
	Libreta Universitaria: 79/11 \\ 
	Correo electr\'onico: leandro.matayoshi@gmail.com \\ 
	Directora: Dra. Luciana Ferrer
	\vspace{10px}
} 


\begin{document}

%----------------------------------------------------------------------------------------
%	TITLE PAGE
%----------------------------------------------------------------------------------------

\begin{titlepage}
\maketitle
\begin{minipage}[t]{\textwidth}
    \begin{minipage}[t]{.55 \textwidth}
        \includegraphics{logo_uba.jpg}
    \end{minipage}%%
\end{minipage}%
\end{titlepage}

%----------------------------------------------------------------------------------------
%	RESUMEN SECTION
%----------------------------------------------------------------------------------------

\section{Resumen}

Ejemplo de cita al paper de Franco Ferrer \cite{franco_ferrer_main_paper} 

La evolución de la tecnología durante estos últimos años ha permitido el desarrollo
de herramientas automáticas que asisten al estudiante durante el aprendizaje de una
lengua extranjera, siendo Duolingo uno de los ejemplos más conocidos de este tipo.

La pronunciación es un aspecto importante a tener en cuenta durante la enseñanza. En la
actualidad, es posible evaluar de forma acertada la pronunciación 
a nivel párrafo u oración, disminuyendo la precisión al 
realizar el análisis para unidades de tiempo más cortas (palabras y fonos). Una correcta 
calificación le permitirá al estudiante conocer acerca de sus potenciales errores 
sobre fonos específicos, de forma tal que pueda trabajar en ellos para corregirlos.

El proyecto dirigido por la Dra. Ferrer, cuyo fin es el desarrollo de un sistema de 
asistencia computarizada para el aprendizaje de inglés de niños Argentinos, constituye
un marco ideal para explorar variantes que permitan mejorar los sistemas de
calificación de pronunciación a nivel fono.

El objetivo principal de la tesis será evaluar la utilización de features que modelan
las dependencias temporales durante la pronunciación de un fono ya que pueden contener
información no utilizada hasta el momento.

%----------------------------------------------------------------------------------------
%	MOTIVACIÓN SECTION
%----------------------------------------------------------------------------------------

\section{Motivación}

El aprendizaje de una lengua extranjera es un proceso complejo que involucra muchos 
aspectos, entre los cuales se encuentra la pronunciación.

La Dra. Ferrer trabaja en un proyecto cuyo objetivo es desrrollar un sistema ACAI 
(Asistencia Computarizada para el Aprendizaje de Idiomas), para niños argentinos 
en proceso de aprendizaje del idioma Inglés. El proyecto cuenta con 3 tres
etapas principales: Recolección de las grabaciones de los infantes y construcción de
la base de datos, el desarrollo de un sistema de reconocimiento automático del habla
para generar los alineamientos temporales y finalmente el desarrollo del sistema
automático de puntuación. La tesis se focalizará en este último punto.

Si bien actualmente es posible obtener estimaciones confiables de la calidad
de pronunciación a nivel párrafo y oración, cuando la meta es generar puntajes 
para unidades más pequeñas como palabras y específicamente fonos, los sistemas del
estado del arte están lejos de alcanzar su máximo rendimiento.

La generación de buenos sistemas de puntaje a nivel fono es un aspecto clave en el marco del
proyecto ya que muchos niños que recién comienzan a estudiar no están en condiciones
de enunciar párrafos u oraciones largas. En suma, identificar errores a este
nivel permite al estudiante focalizar sus esfuerzos en corregir aspectos concretos
y específicos.

%----------------------------------------------------------------------------------------
%	PROPUESTA DE TESIS SECTION
%----------------------------------------------------------------------------------------

\section{Propuesta de tesis}

En este trabajo se continuará con la exploración de soluciones al problema de calificación
de la pronunciación en el marco del proyecto dirigido por la Dra. Ferrer. Nuevamente 
la unidad central de análisis será el fono y los modelos se entrenarán 
a partir de pronunciaciones de personas no nativas, ya que se han logrado buenos resultados al adoptar esta estrategia. 

Replicando y posteriormente utilizando como baseline los sistemas referentes de los trabajos
previos, se desarrollarán clasificadores basados en features dinámicos, no habiéndose 
probado este tipo de features con anterioridad.

%----------------------------------------------------------------------------------------
%	OBJETIVO SECTION
%----------------------------------------------------------------------------------------

\section{Objetivo}

El objetivo es analizar la incorporación de features dinámicos que permiten capturar
las dependencias temporales durante la pronunciación de cada fono.
Los sistemas previos han utilizado en su mayoría características espectrales estándar
en la literatura del procesamiento del habla: \textit{Mel Frecuency Cepstral 
Coefficients}.

Independientemente de si los resultados de los clasificadores mejoren o empeoren, los
experimentos permitirán saber si los features utilizados previamente y los introducidos
en este trabajo proveen información complementaria, lo cual puede ser de gran utilidad
para futuras implementaciones.

%----------------------------------------------------------------------------------------
%	PLAN DE TRABAJO SECTION
%----------------------------------------------------------------------------------------

\section{Plan de trabajo}

\begin{enumerate}
	\item Revisi\'on de la implementaci\'on actual del sistema para calificación de la pronunciación y lectura de la literatura relacionada
	\item Replicaci\'on de experimentos baseline, cuyos features asumen independencia temporal
	\item Exploraci\'on de nuevos sistemas basados en features din\'amicas
	\item Evaluaci\'on de complementariedad de features mediante la fusi\'on de baselines con los nuevos sistemas
	\item An\'alisis de resultados y propuesta de mejoras
\end{enumerate}

\newpage
\printbibliography

\end{document}