%%%%%%%%%%%%%%%%%%%%%%%%%%%%%%%%%%%%%%%%%
% Contract
% LaTeX Template
% Version 1.0 (December 8 2014)
%
% This template has been downloaded from:
% http://www.LaTeXTemplates.com
%
% Original author:
% Brandon Fryslie
% With extensive modifications by:
% Vel (vel@latextemplates.com)
%
% License:
% CC BY-NC-SA 3.0 (http://creativecommons.org/licenses/by-nc-sa/3.0/)
%
% Note:
% If you are using Apple OS X, go into structure.tex and uncomment the font
% specifications for OS X and comment out the default specifications - this will
% drastically increase how good the document looks. You will now need to
% compile with XeLaTeX.
%
%%%%%%%%%%%%%%%%%%%%%%%%%%%%%%%%%%%%%%%%%

\documentclass[a4paper,12pt]{article} % The default font size is 12pt on A4 paper, change to 'usletter' for US Letter paper and adjust margins in structure.tex
% \usepackage[pdftex]{graphicx}
\usepackage[pdftex]{graphicx}

\input{structure.tex} % Input the structure.tex file which specifies the document layout and style

\title{	
\normalfont \normalsize 
\textsc{Universidad de Buenos Aires \\ Facultad de Ciencias Exactas y Naturales \\ Departamento de Computaci\'on} \\ [25pt] % Your university, school and/or department name(s)
\huge Título tema de tesis\\ % The assignment title
\date{\today}
\vspace{20px}
}

\author{
	Alumno: Leandro Matayoshi \\ 
	Libreta Universitaria: 79/11 \\ 
	Correo electr\'onico: leandro.matayoshi@gmail.com \\ 
	Directora: Dra. Luciana Ferrer
	\vspace{10px}
} 


\begin{document}

%----------------------------------------------------------------------------------------
%	TITLE PAGE
%----------------------------------------------------------------------------------------

\begin{titlepage}
\maketitle
\begin{minipage}[t]{\textwidth}
    \begin{minipage}[t]{.55 \textwidth}
        \includegraphics{logo_uba.jpg}
    \end{minipage}%%
\end{minipage}%
\end{titlepage}

%----------------------------------------------------------------------------------------
%	RESUMEN SECTION
%----------------------------------------------------------------------------------------

\section{Resumen}

%----------------------------------------------------------------------------------------
%	MOTIVACIÓN SECTION
%----------------------------------------------------------------------------------------

\section{Motivación}

El aprendizaje de una lengua extranjera es un proceso complejo que involucra muchos 
aspectos, entre los cuales se encuentra la pronunciación.

La Dra. Ferrer trabaja un proyecto de desarrollo de un sistema ACAI (Asistencia
Computarizada para el Aprendizaje de Idiomas), para niños argentinos en en proceso de
aprendizaje del idioma Inglés.

Si bien actualmente es posible obtener estimaciones confiables de la calidad
de pronunciación a nivel párrafo y oración, cuando la meta es generar puntajes 
para unidades más pequeñas como palabras y específicamente fonos, los sistemas del
estado del arte están lejos de alcanzar su máximo rendimiento.

Generar buenos sistemas de puntaje a nivel fono es un aspecto clave en el marco del
proyecto ya que muchos niños que recién comienzan a estudiar no están en condiciones
de enunciar párrafos u oraciones largas. Al mismo tiempo, identificar errores a este
nivel permite al estudiante focalizar sus esfuerzos en corregir aspectos concretos
y específicos.

%----------------------------------------------------------------------------------------
%	PROPUESTA DE TESIS SECTION
%----------------------------------------------------------------------------------------

\section{Propuesta de tesis}

En este trabajo se continuará con la exploración de soluciones al problema de calificación
de la pronunciación en el marco del proyecto dirigido por la Dra. Ferrer. Nuevamente 
la unidad central de análisis será el fono y los modelos se entrenarán 
a partir de pronunciaciones de personas no nativas, ya que se han logrado buenos resultados al adoptar esta estrategia. 

Replicando y posteriormente utilizando como baseline los sistemas referentes de los trabajos
previos, se desarrollarán clasificadores basados en features dinámicos, no habiéndose 
probado este tipo de features con anterioridad.

%----------------------------------------------------------------------------------------
%	OBJETIVO SECTION
%----------------------------------------------------------------------------------------

\section{Objetivo}

El objetivo es analizar la incorporación de features dinámicos a los sistemas, capturando
de esta manera las dependencias temporales de las muestras.

Independientemente de si los resultados de los clasificadores mejoren o empeoren, los
experimentos permitirán saber si los features utilizados previamente y los introducidos
en este trabajo proveen información complementaria, lo cual puede ser de gran utilidad
para planificar futuras estrategias.

%----------------------------------------------------------------------------------------
%	PLAN DE TRABAJO SECTION
%----------------------------------------------------------------------------------------

\section{Plan de trabajo}

\begin{enumerate}
	\item Revisi\'on de la implementaci\'on actual del sistema para calificación de la pronunciación y lectura de la literatura relacionada
	\item Replicaci\'on de experimentos baseline, cuyos features asumen independencia temporal
	\item Exploraci\'on de nuevos sistemas basados en features din\'amicas
	\item Evaluaci\'on de complementariedad de features mediante la fusi\'on de baselines con los nuevos sistemas
	\item An\'alisis de resultados y propuesta de mejoras
\end{enumerate}


\end{document}