% !TEX TS–program = pdflatexmk
%%%%%%%%%%%%%%%%%%%%%%%%%%%%%%%%%%%%%%%%%
% Contract
% LaTeX Template
% Version 1.0 (December 8 2014)
%
% This template has been downloaded from:
% http://www.LaTeXTemplates.com
%
% Original author:
% Brandon Fryslie
% With extensive modifications by:
% Vel (vel@latextemplates.com)
%
% License:
% CC BY-NC-SA 3.0 (http://creativecommons.org/licenses/by-nc-sa/3.0/)
%
% Note:
% If you are using Apple OS X, go into structure.tex and uncomment the font
% specifications for OS X and comment out the default specifications - this will
% drastically increase how good the document looks. You will now need to
% compile with XeLaTeX.
%
%%%%%%%%%%%%%%%%%%%%%%%%%%%%%%%%%%%%%%%%%

\documentclass[a4paper,12pt]{article} % The default font size is 12pt on A4 paper, change to 'usletter' for US Letter paper and adjust margins in structure.tex
\usepackage[utf8]{inputenc}
\usepackage[pdftex]{graphicx}
% \usepackage{biblatex}
\usepackage[
backend=biber,
sorting=none
]{biblatex}
\addbibresource{references.bib}

\input{structure.tex} % Input the structure.tex file which specifies the document layout and style

\title{	
\normalfont \normalsize 
\textsc{Universidad de Buenos Aires \\ Facultad de Ciencias Exactas y Naturales \\ Departamento de Computaci\'on} \\ [25pt] % Your university, school and/or department name(s)
\huge Sistema de calificación de pronunciación a nivel fono para aprendizaje de idiomas
\\ % The assignment title
\date{\today}
\vspace{20px}
}

\author{
	Alumno: Leandro Matayoshi \\ 
	Libreta Universitaria: 79/11 \\ 
	Correo electr\'onico: leandro.matayoshi@gmail.com \\ 
	Directora: Dra. Luciana Ferrer
	\vspace{10px}
} 


\begin{document}

%----------------------------------------------------------------------------------------
%	TITLE PAGE
%----------------------------------------------------------------------------------------

\begin{titlepage}
\maketitle
\vspace{20px}
\begin{minipage}[t]{\textwidth}
    \begin{minipage}[t]{0.55 \textwidth}
        \includegraphics[scale=0.85]{logo_uba.jpg}
    \end{minipage}%%
\end{minipage}%
\end{titlepage}

%----------------------------------------------------------------------------------------
%	RESUMEN SECTION
%----------------------------------------------------------------------------------------

\section{Resumen}

La evolución de la tecnología durante estos últimos años ha permitido el desarrollo
de herramientas automáticas que asisten al estudiante durante el aprendizaje de 
segundos idiomas, siendo Duolingo uno de los ejemplos más conocidos de este tipo.

La pronunciación es un aspecto importante a tener en cuenta durante la enseñanza
de una lengua extranjera. En la
actualidad, es posible evaluar de forma acertada la pronunciación 
a nivel párrafo u oración, disminuyendo la precisión al 
realizarse el análisis para unidades de tiempo más cortas (palabras y fonos). 

El presente trabajo se enmarca dentro de un proyecto más grande dirigido por la Dra. Ferrer,
cuyo fin es el desarrollo de un sistema de asistencia computarizada para el 
aprendizaje de Inglés de niños Argentinos. Este contexto constituye
un escenario ideal para explorar variantes en el desarrollo de sistemas de calificación
de pronunciación a nivel fono, ya que los niños que están dando sus primeros
pasos en el Inglés no tienen la experiencia necesaria para pronunciar párrafos u oraciones
largas.

El objetivo de la tesis es la exploración de nuevas soluciones al problema de calificación
de pronunciación de fonos en segundos idiomas. Para ello, se tomará como punto de partida
el trabajo previo desarrollado por Franco, Ferrer and Bratt 2014 \cite{franco_ferrer_main_paper}.

%----------------------------------------------------------------------------------------
%	MOTIVACIÓN SECTION
%----------------------------------------------------------------------------------------

\section{Motivación}

El aprendizaje de una lengua extranjera es un proceso complejo que involucra muchos 
aspectos, entre los cuales se encuentra la pronunciación.

La Dra. Ferrer trabaja en un proyecto cuyo objetivo es desarrollar un sistema ACAI 
(Asistencia Computarizada para el Aprendizaje de Idiomas), para niños argentinos 
en proceso de aprendizaje del idioma Inglés. El proyecto cuenta con 3 tres
etapas principales: Recolección de las grabaciones de los infantes y construcción de
la base de datos, el desarrollo de un sistema de reconocimiento automático del habla
para generar los alineamientos temporales y finalmente el desarrollo del sistema
automático de puntuación a nivel fono. La tesis se centrará en este último punto.

Si bien actualmente es posible obtener estimaciones confiables de la calidad
de pronunciación a nivel párrafo y oración, cuando la meta es generar puntajes 
para unidades más pequeñas como palabras y específicamente fonos, los sistemas del
estado del arte están lejos de alcanzar su máximo rendimiento.

La generación de buenos sistemas de puntaje a nivel fono es un aspecto clave en el 
desarrollo del sistema ACAI ya que muchos niños que recién comienzan a estudiar 
no están en condiciones de enunciar párrafos u oraciones largas. En suma, identificar 
errores a este nivel permitirá al estudiante focalizar sus esfuerzos en corregir 
aspectos concretos y específicos.

%----------------------------------------------------------------------------------------
%	PROPUESTA DE TESIS SECTION
%----------------------------------------------------------------------------------------

\section{Propuesta de tesis}

La calificación de pronunciación a nivel fono es un área del procesamiento del habla
que no ha sido tan estudiada en comparación con otras. Sin embargo existen
trabajos previos en donde se ha estudiado este tema. 
Entre ellos se destaca Franco, Ferrer and Bratt 2014 \cite{franco_ferrer_main_paper}, 
en donde se analizan 
dos tipos de modelos como posibles soluciones: generativos, mediante \textit{Gaussian 
Mixture Models} y discriminativos mediante SVM (\textit{Support Vector Machines}). 
A pesar de ser diferentes, ambas estrategias producen buenos resultados.

La tarea de desarrollar un sistema ACAI para niños Argentinos en proceso de 
aprendizaje de Inglés ha renovado el interés por las investigaciones llevadas
a cabo en el trabajo de Franco, Ferrer y Bratt 2014 ya que comparten el problema
original: la calificación de pronunciación de fonos en segundos idiomas. Por este motivo
recientemente se ha comenzado una nueva implementación con el objetivo de retomar
los experimentos para los modelos generativos propuestos por el trabajo y explorar 
variantes en búsqueda de mejoras.

Este es el marco en el cual la presente tesis tienen lugar, la cual paralelamente
tomará como punto de partida el modelo discriminativo propuesto en el paper (cuya
implementación no se ha replicado aún) sobre el cual se analizarán 
alternativas no probadas en dicho trabajo. 
Entre ellas se destaca la utilización de features dinámicos que modelen las dependencias
temporales presentes en cada muestra. Esto representa una gran diferencia 
respecto al modelo de referencia, el cual utiliza features provenientes de un proceso
de adaptación de mezclas de Gaussianas entrenandas a partir de características
espectrales estándar en la literatura del habla (\textit{Mel Frecuency Cepstral
Coefficients}).

%----------------------------------------------------------------------------------------
%	OBJETIVO SECTION
%----------------------------------------------------------------------------------------

\section{Objetivo}

El objetivo de la tesis es la exploración de soluciones alternativas al problema de calificación
de pronunciación de fonos en segundos idiomas.

Uno de los puntos en donde se pondrá énfasis es en la utilización de features dinámicas para 
modelar las dependencias temporales.
Independientemente de si los clasificadores propuestos proveen mejor o peor rendimiento
que los existentes, los experimentos permitirán saber si los features utilizados 
previamente y los introducidos en este trabajo aportan información complementaria, 
lo cual puede ser de gran utilidad para futuras implementaciones.
	
%----------------------------------------------------------------------------------------
%	PLAN DE TRABAJO SECTION
%----------------------------------------------------------------------------------------

\section{Plan de trabajo}

La tesis estará guiada por los siguientes pasos:
\begin{enumerate}
	\item Lectura de bibliografía relevante
	\item Replicación de experimentos principales del paper Franco, Ferrer and Bratt 2014 
	\cite{franco_ferrer_main_paper}, utilizando la implementación existente para los modelos
	generativos y agregando el código necesario para el modelo discriminativo. Los mismos servirán como referencia (baselines) para los experimentos posteriores.
	\item Exploraci\'on de nuevos sistemas basados en features din\'amicas
	\item Evaluaci\'on de complementariedad de features mediante la fusi\'on de baselines con los nuevos sistemas
	\item An\'alisis de resultados y propuesta de mejoras
\end{enumerate}

\newpage
\printbibliography

\end{document}