As it was said before, \textit{MFCCs} are the acoustic features chosen for this work.
Even though these raw features extracted from each frame composing the utterances of a phones
would be enough to train a classifier that separates correctly
pronounced and mispronounced phones, a different strategy can be taken. The features of all the
frames of a particular utterance can be gathered together in order to summarize the
general properties of the utterance. This is the purpose of using Legendre Polynomials when
generating the features for the classifier. This technique has been used successfully
in speaker recognition systems \cite{legendre}.

\textit{Legendre functions} are the solutions to \textit{Legendre's differential equation}:

\begin{equation}
(1-x^{2})y''(x)-2xy'(x)+n(n+1)y(x)=0, \ for -1 \leq x \leq 1
\end{equation}

The solution for each particular $n={0, 1, 2 \dotsc} \in \mathbb{N}$ is a polynomial of degree
$n$: $P_{n}(x)$. These solutions are well known for each $n \in \mathbb{N}$, and they can even
be generated recursively. The first few Legendre Polynomials are:

\begin{itemize}
  \item[] $P_{0}(x) = 1$
  \item[] $P_{1}(x) = x$
  \item[] $P_{2}(x) = \frac{1}{2}(3x^{2} - 1)$
  \item[] $P_{3}(x) = \frac{1}{2}(5x^{3} - 3x)$
  \item[] $P_{4}(x) = \frac{1}{8}(35x^{4} - 30x^{2} + 3)$
\end{itemize}

Legendre Polynomials are orthogonal with respect to the $L2 \ norm$ on the interval \mbox{$-1 \leq x \leq 1$}:

\begin{equation}
\int_{-1}^{1} P_{n}(x)*P_{m}(x)dx = 0, \ m \neq n
\end{equation}

Moreover, in the interval \mbox{$-1 \leq x \leq 1$} any function $f$ can be represented as sum of
Legendre Polynomials, leading to the Fourier-Legendre Series:

\begin{equation}
f(x) = \sum_{0}^{\infty}a_{n}P_{n}(x)
\end{equation}

So an infinite terms of Legendre Polynomials over the interval \mbox{$-1 \leq x \leq 1$} can be used
to approximate any function. Each coefficient models a particular aspect of the curve: $a_{0}$
is interpreted as the \textit{mean} of the segment, $a_{1}$ is the slope, $a_{2}$ gives
information about the curvature of the segment and the next coefficients model the fine details.
A particular cutoff $c$ can be chosen in order to limit the number of coefficients of the
Legendre Polynomials to be picked in order to approximate the function in a reasonable manner
while avoiding the complex details that may lead to overfitting.