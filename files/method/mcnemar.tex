\subsection{McNemar's Test} \label{subsection:mcnemar}

\textit{McNemar's Test} is used as an alternative to determine whether the \textit{SVM}
based on the combined features and the \textit{SVM} based only on features
derived from the \textit{GMM} adaptation are significantly different.

\cite{mcnemar} Given a set of $n$ test samples labeled by two systems, $A_{1}$ and $A_{2}$
the results can be resumed in:

\begin{center}
    \begin{tabular}{ | c | c | c | }
    \hline
    & Correctly Predicted by $A_{2}$ & Incorrectly Predicted by $A_{2}$ \\ \hline
    Correctly Predicted by $A_{1}$ & $N_{0,0}$ & $N_{0,1}$ \\ \hline
    Incorrectly Predicted by $A_{1}$ & $N_{1,0}$ & $N_{1,1}$ \\ \hline
    \end{tabular}
\end{center}

\begin{itemize}
\item{$N_{0, 0}$ is the number of samples which $A_{1}$ classifies correctly and $A_{2}$ classifies correctly}
\item{$N_{0, 1}$ is the number of samples which $A_{1}$ classifies correctly and $A_{2}$ classifies incorrectly}
\item{$N_{1, 0}$ is the number of samples which $A_{1}$ classifies incorrectly and $A_{2}$ classifies correctly}
\item{$N_{1, 1}$ is the number of samples which $A_{1}$ classifies incorrectly and $A_{2}$ classifies incorrectly}
\end{itemize}

Let's denote $p_{1}$ the error rate of the system $A_{1}$ and $p_{2}$ the error rate of the
system $A_{2}$. \\
Let's also denote $q_{0,0}$ as the probability of ($A_{1}$ correct $\land$ $A_{2}$ correct),
$q_{0,1}$ the probability \\ of ($A_{1}$ correct $\land$ $A_{2}$ incorrect), etc. Hence:

\begin{multicols}{2}
  \noindent
  \begin{equation}
    % \label{eq:p1}
    p_{1} = q_{1,0} + q_{1,1}
  \end{equation}
  \begin{equation}
    % \label{eq:p2}
    p_{2} = q_{0,1} + q_{1,1}
  \end{equation}
\end{multicols}

The \textit{null hypothesis} $p1=p2$ is equivalent to $q_{1,0}=q_{0,1}$. Defining
$q=q_{1,0}/(q_{1,0}+q_{0,1})$ the \textit{null hypothesis} becomes $q=\frac{1}{2}$.
The probability $q$ is the probability that given that only one of the systems
made an error, it was system $A_{1}$ that did it. Under the \textit{null hypothesis},
$N_{1,0}$ is distributed as a Binomial Distribution $B(n,1/2)$, where $n$ is
the observed number of samples for which only one system made an error: $N_{1,0}+N_{0,1}$.
At this point, a two-tail test can be applied to obtain a \textit{p-value}.

In conclusion \textit{McNemar's Test} is based on cross information by comparing the results of
evaluating both systems on exactly the same sample. \textit{Bootstrapping} differs on this
point because each sample is evaluated independently for each system.
So a complementary analysis using both different techniques can be carried out
in order to determine the statistic significance of the results.

