\chapter*{Resumen}

Los avances tecnológicos de los últimos años permiten la mejora de sistemas
automáticos de asistencia computarizada para el aprendizaje de idiomas. En particular,
un aspecto valioso [...] herramientas que se enfocan en la pronunciación,
detectando errores y asignando puntajes. Dentro de estos últimos, nos interesan
aquellos que califican a nivel fono, ya que
no sólo ponen el foco en errores específicos del estudiante sino que al mismo tiempo
pueden ser utilizados por niños aún incapaces de pronunciar frases
demasiado largas.

A previous work had shown that
Discriminative approach $\rightarrow$
supervectors $\rightarrow$
Adapted GMM Method $\rightarrow$
compared with likelihood ratio based generative approach

In this work
Dynamic Features $\rightarrow$
Temporal information $\rightarrow$
DCT and Legendre Polynomials $\rightarrow$
score combination - features combination $\rightarrow$
phone dependent $\rightarrow$.
Phonetically transcribed Spanish database of around...

Los resultados muestran que para un subconjunto de fonos, la combinación de supervectors
con dynamic features efectivamente reduce el error durante la clasificación, soportando la
hipótesis de que ambos tipos de features contienen información complementaria.
Tanto la combinación a nivel score como a nivel features arrojan resultados similares, por
lo que no es posible destacar una sobre la otra.
