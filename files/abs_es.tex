\chapter*{Resumen}

Los avances tecnológicos de los últimos años han permitido la mejora de sistemas
automáticos de asistencia computarizada para el aprendizaje de idiomas.
Un subgrupo importante de estas herramientas son aquellas que se enfocan en la pronunciación,
detectando errores y asignando puntajes. Dentro de estas últimas, nos interesan
aquellas que califican a nivel fono, ya que
no sólo ponen el foco en errores específicos del estudiante sino que al mismo tiempo
pueden ser utilizados por niños aún incapaces de pronunciar frases
demasiado largas.

% (A previous work had shown that
% Discriminative approach $\rightarrow$
% supervectors $\rightarrow$
% Adapted GMM Method $\rightarrow$
% compared with likelihood ratio based generative approach)

En un trabajo anterior en el área de calificación de pronunciación a nivel fono,
se exploró
un método discriminativo basado en Support Vector Machines (SVM) entrenado
a partir de features llamados \textit{supervectors},
que produce resultados ligeramente mejores a los métodos
generativos comúnmente utilizados en este campo, como el cociente de verosimilitud
entre mezclas Gaussianas (GMMs). Los \textit{supervectors} se obtienen a partir de
un proceso de adaptación de un GMM global entrenado a partir de la totalidad
de las muestras de cada fono.

En este trabajo nos basamos en modelos SVM para
estudiar features dinámicas que modelen las dependencias
temporales de cada muestra. Dos técnicas son analizadas como posibles alternativas:
Polinomios de Legendre y la Transformada Discreta del Coseno (DCT).
El objetivo
es analizar si tanto supervectors como features dinámicas
contienen información complementaria,
a partir de la cual pueda mejorarse la efectividad de los modelos.
Para ello
se proponen dos alternativas: combinación a nivel de features,
para la cual se entrena un único clasificador que discrimine en base a ambos features
y combinación a nivel de puntaje,
para la cual se entrena un clasificador por feature, combinándose ambos puntajes
en el momento de la clasificación.

% In this work
% Dynamic Features $\rightarrow$
% Temporal information $\rightarrow$
% DCT and Legendre Polynomials $\rightarrow$
% score combination - features combination $\rightarrow$
% phone dependent $\rightarrow$.
% Phonetically transcribed Spanish database of around...

Las grabaciones corresponden a 206 hablantes estadounidenses pronunciando distintas
frases en español latino. La base de datos está conformada por 2550 grabaciones alcanzando
un total de 130000 instancias de fonos etiquetadas
por transcriptores profesionales.
Los resultados muestran que para un subconjunto de fonos, la combinación de supervectors
con las features dinámicas efectivamente reduce el error durante la clasificación,
soportando la
hipótesis de que ambos tipos de features contienen información complementaria.
Tanto la combinación a nivel score como a nivel features arrojan resultados similares, por
lo que no es posible destacar una técnica sobre la otra.

\bigskip

\noindent\textbf{Palabras claves:} Asistencia Computarizada para Aprendizaje de Idiomas,
